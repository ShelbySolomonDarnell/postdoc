%\begin{table*}[ht]
\begin{table}[t]
\caption{Potential Courses}
\centering
\begin{tabular}{|p{0.3\linewidth}p{0.2\linewidth}|p{0.48\linewidth}|}
\hline
\textbf{TITLE} & \textbf{OFFERED BY} & \textbf{DESCRIPTION} \\ %&ISO numeric Code\\
\hline
\href{https://ocw.mit.edu/courses/15-075j-statistical-thinking-and-data-analysis-fall-2011/}{Statistical Thinking and Data Analysis} & MITOpenCourseware & Self explanatory \\
\hline
\href{https://ocw.mit.edu/courses/18-s997-high-dimensional-statistics-spring-2015/}{High-Dimensional Statistics} & MITOpenCourseware & Intro to the finite sample analysis of high-dimensional statistical methods, state-of-the-art regression, matrix estimation, and PCA. \\
\hline
Genetics, Genomics \& Informatics Seminar & UTHSC GGI & Discuss state-of-the-art research into genetics and genomics. \\
\hline
\href{https://catalog.uthsc.edu/preview_course_nopop.php?catoid=39&coid=64602}{Foundations of AI in Healthcare I} & UTHSC GGI & ID a biomedical/healthcare area of interest that may benefit from the application of AI/ML. \\
\hline
\href{https://catalog.uthsc.edu/preview_course_nopop.php?catoid=39&coid=64603}{Foundations of AI in Healthcare II} & UTHSC GGI & Explore modification and usage of ML algorithms.\\
\hline
Gene Structure and Function & UAH & Advanced studies of macromolecular structure and biological function of proteins and nucleic acids involved in the passage of genetic information and cellular response. Structural significance of viruses and molecular evolution included. \\
\hline
Biostatistics/AI & UAH with A\&M & \\
\hline
Psychobiology Stress \& Illness & UAH & Overview of physiological stress responses and their influence on health, behavior, and illness.\\
\hline
Bioinformatics I & UAH & Practical use in bioinformatics and X-ray crystallography \\
\hline
Bioinformatics II & UAH & Practical use in bioinformatics and applied genomics.\\
\hline
Microbial Genetics & UAH & Transmission, expression, and evolution of genes in microorganisms. Studies of chromosomes, plasmids, transposons, bacteriophages, and other genetic elements.\\
\hline
Advanced Molecular Techniques & UAH & Laboratory techniques in molecular biology including current methodology in genomics, proteomics, and RNA analysis.\\
\hline
Immunology & UAH & Innate, humoral, and cell-mediated immunity. Immune deficiencies and hypersensitivities. Autoimmunity, transplantation, and other genetic elements.\\
\hline
\end{tabular}
\end{table}

%\begin{table*}[ht]
\begin{table}[t]
\caption{Potential Courses}
\centering
\begin{tabular}{|p{0.3\linewidth}p{0.2\linewidth}|p{0.48\linewidth}|}
%\multicolumn{3}{|c|}{\textbf{Possible Courses}} \\
\hline
\textbf{TITLE} & \textbf{OFFERED BY} & \textbf{DESCRIPTION} \\ %&ISO numeric Code\\
\hline
Algorithms in Bioinformatics & UAB & This course introduces various fundamental algorithms and computational concepts for solving questions in bioinformatics and functional genomics. These include graph algorithms, dynamic programming, combinatorial algorithms, randomized algorithms, pattern matching, classification and clustering algorithms, hidden Markov models and more. Each concept will be introduced in the context of a concrete biological or genomic application. A broad range of topics will be covered, ranging from genome annotation, genome reconstruction, microarray data analysis, phylogeny reconstruction, sequence alignments, to variant detection. \\
\hline
Next-generation Sequencing Data Analysis & UAB & This course is aimed to equip participants with the essential knowledge and skills required to begin analyzing next-generation sequencing data and carry out some of the most common types of analysis. The topics covered in-depth during this course are the analysis of RNA-Seq, ChIP-Seq data, ATACseq data, and Single-cell data, with an optional Variant Calling session. The sessions will also include Introduction to next-generation sequencing (NGS) technologies, common NGS data analysis issues, applications of sequencing technologies, introduction to bioinformatics file formats (e.g. FASTQ, bam, bed) and bioinformatics toolkits. At the end of this course, participants will have the expertise to perform these data analysis independently.\\
\hline
Visual Analytics for Bioinformatics & UAB & We will cover representation of different data types, concentrating on those generated by data-rich platforms such as next-generation sequencing applications, flow/mass cytometry, and proteomics, and will discuss the use of visualization techniques applied to assessing data quality and troubleshooting. \\
\hline
\href{https://www.coursera.org/specializations/bioinformatics?action=enroll}{Bioinformatics specialization} & UC San Diego via Coursera & Master bioinformatics software and computational approaches in modern biology.\\
\hline
\end{tabular}
\end{table}