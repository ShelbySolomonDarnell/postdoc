\section*{Introduction} % The \section*{} command stops section numbering

\addcontentsline{toc}{section}{Introduction} % Adds this section to the table of contents

My community is medically under-served and statistically underrepresented in the (health) sciences. 
Recently I started attending a meeting with an NSF project run by Prins a.o. to build a special pangenome supercomputer in collaboration with Cornell. 
Unsurprisingly, with 25 people attending, I am the only black person in the room.
At a personal level, my young and non-obese sister has lost both kidneys because of complications from diabetes and hypertension, and I constantly ask myself `why' or what are the causal factors leading to such disaster.
Which further leads me to keep track of my own susceptibility to the same health issues, in order to avoid a similar disaster, while asking `what could she have done differently to avoid the cataclysm?'
What can I do with my computer science background to support designer medicine for anyone who needs it.  
These questions led me to engage Pjotr Prins and my interest in pursuing research in pangenomes.

According to HudsonAlpha, `Pangenomes present all of the genes and DNA sequences within a species'. 
At the beginning of research and application of genetic information, once the human genome was fully sequenced, a single persons genome was used as the reference genome due to the time and expense of genetic sequencing.
These reference genomes are used as comparators for disease or trait causes DNA changes and identifiers, cross species similarity, and other aspects of genomic research.
When considering that certain disease and health issues are related to specific populations and different groups and people have different susceptibility to illness, having a single individuals genome representing all of humanity, the quality of the discovery is as limited as the diversity in a single person's genetic information.
Hence the firm push, with funding for the development of as diverse a pangenome as possible, and the development of proper tools to support genomic research. 
Sucn a pursuit leads us to the human genome project.


The \href{https://humanpangenome.org/about-us/}{human genome project}, funded by the \textbf{N}ational \textbf{H}uman \textbf{G}enome \textbf{R}esearch \textbf{I}nstitute (NHGRI), is pursuing many different avenues to meet their goal, of creating a pangenome that represents the world's diversity.  
Thankfully researchers at Cornell and UTHSC are developing a specialized computer with optimized algorithms for exploring pangenomes.
Many avenues are being pursued while Cornell leads researchers toward this goal, some of which are: privacy preserving pangenomics, pangenome visualization, and broader impacts.
Supporting the development of a pangenome computer and softwares will enable the development of optimzed designer medicine and therapeutics in the future; hence, my excitement of pursuing this change in my career focus.
These topics are mentioned as I intend to contribute to these and expand of the same in this proposal.


%As a postdoc working with the genetics, genomics and informatics (GGI) department at UTHSC and a computing specialist I want to focus my efforts on pointing my research trajectory to the answers of the aforementioned questions, and enabling others to accomplish the same with the tools created to enable my goals.

%Causal inference, a new aspect of computer science and artificial intelligence pioneered by Turing award winner Judea Pearl\cite{Pearl:1995, Pearl:2009, Pearl:2018, Pearl:2019} and his collaborators, is a tool that enables one to use data, mathematics, and his own structural causal model to answer complicated questions hereto for unanswerable using pre-existing AI/ML methods.
%However, before being able to apply such ground-breaking techniques to answer my questions; I must do the following:
%\begin{itemize}[noitemsep]
%    \item attend seminars and classes on Genetics and Genomics
%    \item augment existing tools for \GN\
%    \item understand and practice do-Calculus and structural causal models
%    \item properly acquire the data needed to begin answering the questions of interest
%\end{itemize}

In addition to working on topics about which I am passionate, one of the main purposes of a postdoc is to enable the all-around professional growth of a young researcher.
Hence the results of an individual development plan will be combined with my research interests to determine the outcomes of this postdoctoral research.