\subsection{Panorama Project}

The Panorama project is a five year NSF funded effort to create the first integrated rack scale acceleration paradigm specifically for computational pangenomics.
Christopher Batten and a team of seven primary investigators, including Pjotr Prins, currently lead this effort.

\subsubsection{Problem}
It has become necessary for computers to attempt analysis of massive datasets which need to be manipulated in irregular and rapidly changing ways while ensuring strict privacy guarantees.  
The ability to efficiently support large, sparse, dynamic yet private data for solving big complex problems on hetergeneous systems is one of the grand challenges in software/hardware systems research.
Addressing this grand challenge is the Panarama project by way of the exploration of integrated rack scale acceleration for computational genomics.
Integrated rack-scale acceleration refers to an emerging computer-systems paradigm that uses tens of tightly integrated computing nodes, each of which includes a mix of general-purpose processors and specialized accelerators interconnected with a special-purpose network. 
Computational pangenomics refers to a recent trend towards representing genomes, the genetic material of an organism, not as a single linear sequence of DNA base pairs but instead as an intricate network of sequences that efficiently represents the relationships between many individuals' genomes at once. 
Computational pangenomics naturally captures the trend towards big, sparse, dynamic, and private data and is thus a perfect application domain to explore heterogeneous software/hardware systems research. 

\subsubsection{Novelties}
The project's novelties are: a truly cross-stack approach spanning applications, programming languages, compilers, architecture, security, and privacy including use of a one-of-a-kind Panorama prototype system; new hardware techniques to accelerate domain-specific computing and to unify heterogeneous systems; new software techniques to let programmers harness the performance advantages of heterogeneous systems; and new software/hardware techniques to make such heterogeneous systems more secure.

\subsubsection{Impacts}
The project's impacts are: to specifically enable computational biologists to better see the ``genetic dark matter'' of population-wide genomics which has been to date hidden, opening up new scientific discoveries; and to more generally enable future computer users to more easily take advantage of heterogeneous computer systems to solve large and complex problems. 
This project is also pursuing two broader impact initiatives. 
The first is an ambitious yet concrete initiative to increase participation of under-represented minority students in computer science by developing a low-level computer-systems module for a new four-week summer program targeting rising sophomores. 
The second involves specific plans to grow the open-source software/hardware ecosystem in the computational-biology and computer-systems communities.

\subsubsection{Methods/Project Thrusts}
The Panorama project includes a highly interdisciplinary team of researchers across four focus areas: applications (computational biology), programming languages \& compilers, computer architecture, and security \& privacy. 
The team is taking a holistic software/hardware co-design approach to explore five tightly interconnected research thrusts. 
The first three thrusts are structured from top-down across the computing stack. 
Thrust 1 investigates new computational pangenomics data structures and algorithms and will develop PanoBench, a new benchmark suite suitable for driving the remaining thrusts. 
Thrust 2 investigates new programming-language and compiler techniques \cite{umar2022, xiang2022, hua2022mgx}. 
Thrust 3 investigates new computer architectures with support for a whole-rack manycore with 1M+ cores and a partitioned global address space, unified array-based accelerators, and application-specific accelerator chiplets for computational pangenomics. 
The final two thrusts cut across both software and hardware. 
Thrust 4 investigates new security and privacy techniques including scalable secure computation on heterogeneous rack-scale systems, secure rack-scale resource management with auto-tuning, and differential privacy and homomorphic encryption for pangenomics. 
Thrust 5 involves holistically evaluating the research ideas in the other thrusts through the use of a one-of-a-kind Panorama prototype system.


\subsection{Differential Privacy for Pangenomes}
We implement a differentially private haplotype sampling method in a pangenomics toolkit. 
It projects $\epsilon$-differentially private synthetic pangenome variation graphs out of pangenomes built from complete haplotype-resolved assemblies like those made in the Human Pangenome project. 
We generate $\epsilon$-differentially private graphs from the human \textit{m}ajor \textit{h}istocompatibility \textit{c}omplex (MHC), and use these to explore the effects of algorithm paramaters on output.

Large medical cohorts with associations between genomes and phenotypes usually only provide controlled data access to trusted researchers. Our goal is to establish a standard whereby a they could release a fully-public transformation of this controlled data for global use by anyone. This would thus provide global biomedical utility without significant risk to study participants. We do so by applying tools from differential privacy.

Differential privacy is an approach to data release that which allows for the description of group characteristics without revealing information about single individuals. It quantifies privacy loss caused by the release of information, allowing us to reason about the risks that a particular data sharing model poses to individual privacy. We build on decades of work on differential privacy, which has yielded well-defined models of differentially private data publication \cite{dwork2014}. Our work is similar to approaches that have been used for trajectory data release, but differs in the unique biomedical context and properties of the graph data structures we use. This has led to the need for an application-specific implementation.

Differential privacy provides tools that allow us to generate privacy-preserving synthetic databases out of real ones. We pursue this approach because we judge that it is easier to share static databases than organize access to differentially private queries. If we provide a tool to produce a differentially private synthetic pangenome, a genomics research project could release a pangenome that can be reused indefinitely by external researchers. Reuse of this database would not constitute increased privacy risk to individuals \cite{dwork2014}.
