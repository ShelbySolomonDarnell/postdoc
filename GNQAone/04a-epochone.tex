\subsection{Optimizing an LLM for research that uses the GeneNetwork Genomics database}

As recognized by other researchers, specifically \cite{Tan:2017}, data acquisition is one of the most important aspects of any expert system. 
Before building every component of the expert system we will test and optimize the major components one at a time.
By utilizing existing tools and LLM's we will load hundreds of documents with an overall specified area on which we want the system to be an expert.
This work focuses on the Web QTL or GeneNetwork.org genomic database containing many terabytes of rat and mouse genomic information, as referenced by many \cite{Chesler:2004,Alberts:2010} \cite{Alberts:2010} \cite{Mulligan:2017} \cite{Watson:2020}.

\subsubsection{How it's going}
At the moment oracles have not been added to the system, we have submitted 950 publications to tune an LLM.
All of the articles are either about GeneNetwork.org or use its data.
The system uses the following models: [ask Adrian to provide the information].

A biologist has been able to test the system and asked it a range of questions.
Their aim was to get a comprehensive understanding of what the types of questions the system ``should'' be able to answer.
The investigators' questions and comments come from their frame of reference, specifically whether or not the system is answering questions correctly and whether the response, if not a direct answer, will be able to lead one to the truth.
%, then compared its responses to asking \openai's GPT3.5 model directly.
Summarization and inference abilities were tested through the investigators questions.
A full transcription of the questions and the investigators comments can be found in the appendix.% add a reference
This system needs to show more specific knowledge about genomics research with respect to rats and mice, as the same is the focus of the data on GeneNetwork.org.

%Shelby Solomon Darnell will have to run drills using workman.