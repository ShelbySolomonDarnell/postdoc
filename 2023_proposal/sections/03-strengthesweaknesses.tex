Science careers has a tool to aide young researchers in building an individual development plan, called
\hyperlink{https://myidp.sciencecareers.org}{myIDP}.

\subsection{Strong skills}
\begin{enumerate}[noitemsep]
    \item basic writing and editing
    \item writing for nonscientists
    \item speaking clearly and effectively
    \item presenting to nonscientists
    \item demonstrating workplace etiquette
    \item complying with rules and regulations
    \item maintaining positive relationships with colleauges
    \item providing instruction and guidance
    \item creating vision and goals
    \item serving as a role model
    \item careful recordkeeping practices
    \item understanding of data ownership/sharing issues
    \item demonstrating responsible authorship and publication practices
    \item demonstrating responsible conduct in human research
    \item how to maintain a professional network
    \item technical skills related to my specific research area
\end{enumerate}


\subsection{Improvement Approach}

\subsubsection{Areas for Improvement}
\begin{enumerate}[noitemsep]
    \item writing grant proposals
    \item developing/managing budgets
    \item delegating responsibilities
    \item planning and organizing projects
    \item how to negotiate
    \item contributing to institution (e.g. participate on committees)
    \item demonstrating responsible conduct in animal research
    \item statistical analysis
    \item seeking advice from advisors and mentors
    \item how to interview
    \item navigating the peer review process
\end{enumerate}



Write grant proposals for NSF Experiential Learning for Emerging and Novel Technologies (ExLENT) and the NIH Diversity Supplement opportunities.
Each will give experience with programs that target diverse populations and support investigations into artificial intelligence techniques.
In working on the preparation of these initial grant proposals, more than one area of improvement will be worked on, namely developing budgets, and planning projects.

Before writing the grant proposals research work must be completed up to a point of having published and presented some research work.
During the research I will work on seeking advice from advisors and mentors during the work, and learn how to better navigate the peer review process after submission of manuscripts.


Negotiation is another week point in my experience, to this end I will have to learn from my mentors negotiation tactics as I build my research team and interact with collaborators.

%One of the weakpoints listed in myIDP evaluation is that I have not demonstrated responsible conduct in animal research, that is because 
I find when uncomfortable friction appears in professional settings I do not negotiate the situation well, as in I copitulate to the aggressive party to avoid the friction.
I must learn how to navigate the friction and negotiate the best outcome for myself and my team, as the best outcome will push forward my major goals.
An activity I want to avoid, but should not -based on Pjotr's feedback- is planning/organizing events.
From my experience such an activity requires interacting with myriad types of personalities, many of whom I will have little common ground.
In order to become a well-known researcher I will be required to organize conferences, symposiums, research experience for undergraduates programs and more.
I will ask my mentors to be on the lookout for events I can help organize. 

This schedule includes learning activities, speaking engagements, research, writing, collaboration, and working on specific weak skills and deficiencies.
With this schedule mapped out to improve my overall research scientist qualities, the following list shows the main research thrusts of the postdoc.

\subsubsection{Training in Biology and Genomics}
\begin{description}
    \item[1. Scholarly Activities] The primary emphasis of this opportunity is to apply a comprehensive AI skillset and my diverse perspective to Panorama project. With Drs. Prins, Batten and Garrison serving as my mentors I will be guided properly as to developing solutions to meet the research requirements.
    \item[2. Educational Activities] New professors are encouraged to participate in seminars that help keep each other on the cutting edge of the field. As a computer scientist getting into genomics I will take several classes to enhance my skills. I will also attend the genomics seminar to keep abreast of the cutting edge in the field, while strengthening my core competencies.
    \item[3. Speaking Activities] There are many conferences in the field of genomics with artificial intelligence and data science tracks. Two upcoming such conferences with submissions due on July 31, 2023 are: the International Conference on Genetics and Genomics, the International Conference on Quantitative Genetics and Genomics \href{https://waset.org/quantitative-genetics-and-genomics-conference}{(ICQGG)}, and the International Conference on Computational Genomics and Evolution \href{https://waset.org/computational-genomics-and-evolution-conference}{(ICCGE)}. 
  \end{description}

Concerning the research component of the plan, three major aims will be 
\begin{enumerate}[noitemsep]
    \item Differential privacy algorithm development and testing
	\item Aide in development of new pangenome layout algorithms
	\item Support broader impact initiatives 
\end{enumerate} 

Each of these research components will be expounded upon in an upcoming section.