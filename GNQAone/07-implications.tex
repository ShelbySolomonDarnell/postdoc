\section{Implications}
By bringing all AI technological advancements to the fore for genetics and genomics research results in an expert system that we expect to be a strong boon to research.
\project\ utilizes best practices fom different fields, including human-cnetered computing and artificial intelligence, to not only create an expert system to support the pursuit of translational genomic research but also to support the expansion of a benchmark on how to rate the performance of an LLM-based expert system.
Upon developing the system to meet the expectations of biologists, to the extent such is possible, we will have a system that can be focused on different topics and source material.
Source material, as we reference it, are datasets, reference functional annotations, and multiple metabases, databases, knowledge bases with respect to a single topic.
By plugging in all known and available data about a topic into a single interface that is queried using normal human language should accelerate scientific advancement.
Our \project\ is being built as open source software, and will soon use open source \llms, using \guix\.
\guix\ development will allow anyone to quickly and completely recreate the development environment we implement for development and deployment, making this type of system available to anyone who has interest.
We are building this \project\ in hopes it can stand as an example of how to have FAIR system, as the NIH requests from scientists.