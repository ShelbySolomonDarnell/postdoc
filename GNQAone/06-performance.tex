\section{Evaluation \& Performance}

The system is a complex question and answer system, as defined by \cite{Daull:2023:complex}.
The performance of such a system needs to be measured quantitatively and qualitatively.
Qualitative measures such as usability and usefulness seem obvious and not necessary to evaluate, as it is a \gpts\ type of system, where such systems have a question answering functionality and has seen enormous use since its popular debut in late 2022.
However, the \project\ is an expert system for a very specific domain, which will act as a research companion.
To this end the normal qualitative performance factors, e.g. the system usability scale as described by \cite{Blattgerste:2022}, must be evaluated for this system and domain.

%Speak to the qualitative and quantitative shortcomings of the system as it exists.
%Some shortcomings are in the GUI and others have to do with how the model is tuned, and now there are some improvemets our collaborators need to make.

As the work moves on system performance will be measured similarly to \cite{Daull:2023:complex}, by using:
\begin{itemize}
    \item speed of inference
    \item multi-document summarization
    \item biases reduction
    \item how answers compare to large non-specilized \llms\
    \item Usefulness for biologists
    \item Usefulness for non-biologists
    \item usability
    \item ease-of-use
    \item safety and data sensitivity, truthfulness, veracity and hallucinations
    \item alignment to human intentions, expectations and values
     %and receiver operating characteristic (ROC) curve, .
\end{itemize}

Some performance measures like, speed of inference and multi-document summarizaiton and biases reduction, will not be evaluated, until w begin using an open source model hosted at our institution for training.
At the moment we are fine-tuning an OpenAI model.
OpenAI models, as their models have been the main generators of excitement for large language models, will be treated as the default benchmark against which future work will be compared.

Our proof-of-concept system has been tested by a subject matter expert, and his comments on the system's ability to answer questions in fine-tuned domain show us several improvement areas, including: knowledge of specific genes, the answer being pulled from a single source, concerns with hallucination, the breadth of a summary given the references from which it is said to come, retaining a working knowledge of previous queries, responses not as strong as from its foundation model and general core knowledge of research area.

Some of the biologists comments that question the systems knowledge follow: 
\begin{quotation}
\cyan{What is the most recent paper published using the BXD population?}

\textbf{The most recent paper published using the BXD population is from 2018. The most recent paper published using the BXD population is from 2019.} 

\textit{I am assuming the two different responses (some days apart) are because the database of papers is being actively expanded. I can’t actually verify the LLMs statement as the citation information is not exposed yet (just a hash). The top hit quotes from David’s ‘expanded BXD family’ paper from 2019, so it seems OK. Not sure why there is more than one hit for this query though (is the LLM actually collecting dates and finding the max? Or is it just making up text?)}

\cyan{How many strains are available for the BXD population?}

\textbf{There are a totoal of 123 BXD strains available.} 

\textit{I have received some different values, but all under the 150 quoted in David's 2019 paper (which we know from above is in the reading list).}
\end{quotation}

\llms\ are known to have problems with answering questions like `What is the time?', when there is no `oracle/expert' present to provide the precise value.
The disappointment with the system not being completely correct on specific facts that may be found in the uploaded literature is understandable and fixable with up-to-date information and an `oracle/expert' that can provide the answer.

% the same is modified to be different than the model trained with the domain information.
Based on feedback (table~\ref{tab:rupertsfeedbackp1} and table~\ref{tab:rupertsfeedbackp2}) from a biologist we find the more pertinent performance factors include: alignment to human intentions, expectations and values, along with comparison to non-specialized \llms\, and usefulness for biologists.
A measure like biases reduction must be properly defined for it to have meaning.
We build this expert system with a specific community in mind; hence, the biases toward genetics, genomics should be high.
However, as is commonly found in research, researchers study groups with which they have the most experience, which leads to diversity biases, as the genetics and genomics field is most represented by those of European and East Asian descent.
Medical research suffers from non-conscious stereotyping and prejudice, a most persistent legacy of slavery and Jim Crow in the United States of America~\cite{Stone:2011}.
Hence, for this system we must provide a caveat to users that speaks to the history of medical research, and then present facts on which the system will be fine-tuned.
    

After obtaining feedback from multiple biologists familiar with \GN\ and its ecosystem a list of questions will be developed that is representative of what a biologist thinks the system should perform well. 
The same will be turned into a digital survey, similar to work by \cite{Blattgerste:2022} that can be taken from a link on the site when the system is ready for further analysis.

Once we move into phase three for the development of \project\ the question of usefulness will be the most important to answer.
When functonality is added, it is best to design it to supplement and support the \project\ and not overshadow it.
\GN\ is a large database that contains data from multiple decades of research that has been used for research experimentation.
When one asks about a specific experiment of class of experiments, being able to recreate the digital environment that supported the analysis is of paramount importance, and brings us back to the benefits of \guix.
The \project\ should be able to give instructions and links to data and software, along with instructions on how to recreate a study and links to data, metadata and other findings.

In addition to the list of performance metrics above, we look to Standford University's holistic evaluation for language models (HELM) benchmark as described by \cite{Liang:2022:helm}.
\llms\ are poised to replace regular internet `search', as evidenced by Microsoft's bing chat~\cite{BingChat:2023} combine their search functionality with their proprietary LLM to provide a more verbose and helpful interface than either service provides alone.
HELM is being looked at because it takes a human-centered approach to LLM evaluation, which we want to follow as we work in the field of translational medicine and ar building a platform that should be trustworthy and helpful to people by design, not just correct.
\llms\ are being implemented so quickly that there are no regulations made to protect against misuse of the technology, and the system we are building should be an example of \llms\ being used correctly for the broadest imaginable swath of people and use cases, while being malleable to conform to different cultural norms and use cases on the edge.

Evaluating \project\ will require a modification of HELM, as the issues of greatest importance to the operation of this expert system are scientist/human-centered perspectives of operation and overall correctness.
Our expert system needs to support research, and grease the wheels of solutioning.

%This list will be tested among multiple system configurations and wth different platforms to come up with false positives, true positives, and other categories that will allow us to use the ROC curve.

