Shelby Solomon's individual development plan is based on the strengths and weaknesses listed in section~\ref{sec:weaknesses} and will build off of the background~\ref{sec:background}.

\subsection{Technical \& Research Aims}

\begin{description}
	\item[Aim] Differential privacy algorithm development and testing
	\item[Concept] To enable allowing a human pangenome graph to be constructed and publicly released using genome data from a diverse set of genome data by ensuring strong privacy for individuals.
	\item[Idea] Work with Garrison on securing individual and group data with DP algorithms and integrate into PanoBench. Apply more artificial intelligence techniques to testing the efficacy of the already implemented algorithms.
\end{description}

\begin{description}
	\item[Aim] Aide in development of new pangenome layout algorithms
	\item[Concept] These layout algorithms can provide effective visualization that reveals the detailed structure of regions of the human pangenome, which were completely invisible to genomics researchers before.
	\item[Idea] Zhang and Garrison use stochastic gradient descent, a type of AI optimization with which I would love to experiment.
\end{description}

\begin{description}
	\item[Aim] Support broader impact initiatives 
	\item[Concept] All NSF grants must contain broader impact initiatives. This is due to the nature of work being funded by the taxpayer, it follows that projects should have as a partial focus `broader impacts' or aspects that further things that are `good for the people'.
	\item[Idea] I have participated in many `broader impacts' initiatives and guided many underrepresented students in the computing sciences, and look forward to supporting the same for the Panorama project by aiding in the mentoring and management of the low-level computer systems module for the 4-week summer program and support/grow the OSS ecosystem in computational biology.
\end{description}

\subsection{Mentoring Plan}

\subsubsection{Mentorship Team}
My mentorship team consists of Pjotr Prins, Erik Garrison and Christopher Batten. Pjotr's mentoring responsibilities include general software development leadership (Aims 1 \& 2), open source software contribution (broader impacts Aim 3), and genetics/genomics guidance.  
I intend to work very closely with Erik's team as they are building the tools for which I see the application of artificial intelligence techniques being applicable.
Christopher will mentor me for Aim 3, supporting broader impacts by including diverse undergraduate student in research.

\subsubsection{Meeting and Evaluation Plan}
I will have bi-weekly meetings with Pjotr either in person on online. 
Whereas I will be working closely with Erik's team and have almost daily meetings, while also scheduling bi-weekly evaluation meetings with Erik one-on-one.
Christopher holds a weekly meeting about project progress that I will attend regularly, and give scheduled talks with updates about the progress of my work.

\subsubsection{Training in Genomics}
As a computer science PhD genetics, genomics and pangenomics are new areas to me.
As such I will participate in educational activities, aka course work to support the research. 
The potential courses are listed in the appendix.

\subsection{Postdoc timeline}
\captionsetup{font={scriptsize,sc,up,singlespacing}}
\begin{figure*}[h] % Figure at bottom of the page ([b] argument, could be "t" for top or "h" for here)
	\centering
	\includegraphics[width=\textwidth]{Figures/timeline-postdoc.eps}
	\caption{\textbf{Postdoc Timeline}\\
	* - Black Data Processing Associates, bdpa.org \\
	** - ACM/CMD-IT Tapia Celebration of Diversity in Computing\\
	*** - International Conference on Quantitative Genetics and Genomics\\
	**** - International Conference on Computational Genomics and Evolution
	}
	\label{fig:timeline}
\end{figure*}


%\includegraphics[width=0.4\textwidth, angle=0]{image1.pdf}